% $date: 2017-01-09
% $timetable:
%   g1011r2:
%     2017-01-09:
%       1:

\section*{Оценочные задачи}

% $authors:
% - Антон Сергеевич Гусев

\begin{problems}

\item
В~магазине в~ряд висят $21$ белая и~$21$ фиолетовая рубашка.
Найдите такое минимальное~$k$, что при любом изначальном порядке рубашек
можно снять $k$ белых и~$k$ фиолетовых рубашек так, чтобы оставшиеся белые
рубашки висели подряд и~оставшиеся фиолетовые рубашки тоже висели
подряд.

\item
На~окружности отмечено $2 n - 1$ точек ($n \geq 3$), причем ровно $k$ из~них
черные.
При каком минимальном $k$ можно утверждать, что всегда найдется такая пара
черных точек, что на~дуге, соединяющей эти точки, будет лежать ровно
$n$ отмеченных точек?
% граф разбивается на циклы, зависит от~делимости на 3.

\item
На~экзамен пришло несколько школьников, каждый из~которых вытянул один билет
с~номером от~$1$ до~$30$.
Экзаменатор может зачитать список из~нескольких номеров (возможно~--- одного)
и~попросить поднять руки владельцев соответствующих билетов.
За~какое минимальное число таких действий экзаменатор сможет разобраться, кому
какой билет достался?

\item
Безумный танкист (неподвижная точка плоскости) угрожает всех уничтожить,
а~отряд комсомольцев пытается огородить его бетонными стенами
(непересекающиеся отрезки на~плоскости).
Снаряд танка пробивает $k$~стен, но~застревает в~$(k + 1)$-ой.
Какое минимальное количество стен потребуется, чтобы вне зависимости от~выбора
танкистом направления стрельбы его снаряд застревал в~одной из~стен?

\item
В~детской выездной школе после отбоя вожатый пытается поймать нарушителя
спокойствия.
Корпус лагеря состоит из~$n$ комнат, расположенных в~ряд.
Каждую минуту вожатый проверяет одну из~комнат на~предмет наличия в~ней
нарушителя.
После того, как вожатый покидает комнату, нарушитель мгновенно через окно
перебирается в~одну из~соседних комнат (нельзя оставаться на~месте).
Ни~начальное положение, ни~перемещения нарушителя вожатому не~известны.
За~какое минимальное время вожатый сможет гарантированно поймать нарушителя?

\item
В~таблице $10 \times 10$ расставлены числа от~$1$ до~$100$:
в~первой строчке~--- от~$1$ до~$10$ слева направо,
во~второй~--- от~$11$ до~$20$ слева направо и~т.\,д.
Василий разрезает таблицу на~доминошки и~складывает все $50$ произведений пар
чисел в~доминошках.
Он стремится получить как можно меньшую сумму.
Как ему следует разрезать таблицу?

\item
На~доске выписаны числа $1, 2, 4, \ldots, 2^{2015}$.
За~ход разрешается выбрать два из~них, стереть и~записать вместо них их
полусумму.
В~каком порядке следует производить операции, так чтобы оставшееся в~самом
конце число было как можно больше?
% Ключевое утверждение: последний ход должен быть в~последнее число.
% Довольно техническая.

\end{problems}

