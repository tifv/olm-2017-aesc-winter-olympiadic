% $date: 2017-01-11
% $timetable:
%   g9:
%     2017-01-11:
%       2:

\section*{Логика}

% $authors:
% - Виктория Владимировна Журавлёва

\subsection*{Тест}

На острове живут рыцари, говорящие всегда правду, и лжецы, которые всегда лгут.

\begin{enumerate}

\item
$B$ сказал $A$: <<Ты лжец>>.
Какие из следующих утверждений истинны:

\begin{enumerate}

\item <<$A$ --- лжец>>

\item <<$B$ --- лжец>>

\item <<$A$ и $B$ разных типов>>

\item <<$B$ --- рыцарь, $A$ --- лжец>>

\item <<Если $A$ ---  рыцарь, то $B$ --- лжец>>

\item <<Если $B$ ---  лжец, то $A$ --- рыцарь>>?

\end{enumerate}

\item
Известно, что если $A$ --- рыцарь, то $B$ --- рыцарь.
Какие из следующих утверждений заведомо верны:

\begin{enumerate}

\item <<$A$ --- рыцарь>>

\item <<$A$ и $B$ одного типа>>

\item <<Если $A$ ---  лжец, то $B$ --- лжец>>

\item <<Если $B$ ---  рыцарь, то $A$ --- рыцарь>>

\item <<Если $B$ ---  лжец, то $A$ --- лжец>>?

\end{enumerate}

\item
$A$ сказал: <<Если я лжец, то и $B$ --- лжец>>.
Какие из следующих утверждений заведомо верны:

\begin{enumerate}

\item <<Если $A$ ---  лжец, то $B$ --- лжец>>

\item <<$A$ --- лжец>>

\item <<Если $A$ ---  рыцарь, то $B$ --- лжец>>

\item <<Если $A$ ---  рыцарь, то $B$ --- рыцарь>>

\item <<$A$ и $B$ --- лжецы>>

\item <<$A$ и $B$ одного типа>>

\item <<Если $B$ ---  рыцарь, то $A$ --- рыцарь>>

\item <<Если $A$ ---  рыцарь, то о $B$ ничего нельзя сказать>>

\item <<Если $A$ ---  лжец, то $B$ --- рыцарь>>

\item <<Если $A$ ---  рыцарь или $B$ --- лжец>>

\end{enumerate}

\end{enumerate}


\subsection*{Основные задачи}

\begin{problems}

\item
За~круглым столом сидят 10 человек, каждый из~которых либо рыцарь, который
всегда говорит правду, либо лжец, который всегда лжет.
Двое из~них заявили: <<Оба моих соседа~--- лжецы>>,
а~остальные восемь заявили: <<Оба моих соседа – рыцари>>.
Сколько рыцарей могло быть среди этих 10~человек?

\item
За~круглым столом сидят 30~человек~--- рыцари и~лжецы
(рыцари всегда говорят правду, а~лжецы всегда лгут).
Известно, что у~каждого из~них за~этим~же столом есть ровно один друг, причем
у~рыцаря этот друг~--- лжец, а~у~лжеца этот друг~--- рыцарь
(дружба всегда взаимна).
На~вопрос <<Сидит~ли рядом с~вами ваш друг?>> сидевшие через одного ответили
<<Да>>.
Сколько из~остальных могли также ответить <<Да>>?

\item
Тридцать девочек~--- 13 в~красных платьях и~17 в~синих платьях~--- водили
хоровод вокруг новогодней елки.
Впоследствии каждую из~них спросили, была~ли ее соседка справа в~синем платье.
Оказалось, что правильно ответили те и~только те девочки, которые стояли между
девочками в~платьях одного цвета.
Сколько девочек могли ответить утвердительно?

\item
На~совместной конференции партий лжецов и~правдолюбов в~президиум было избрано
32~человека, которых рассадили в~четыре ряда по~8~человек.
В~перерыве каждый член президиума заявил, что среди его соседей есть
представители обеих партий.
Известно, что лжецы всегда лгут, а~правдолюбы всегда говорят правду.
При каком наименьшем числе лжецов в~президиуме возможна описанная ситуация?
(Два члена президиума являются соседями, если один из~них сидит слева, справа,
спереди или сзади от~другого.)

\item
На~11 листках бумаги написаны 11~фраз (по~одной на~листке):
\\
\emph {(1)} Левее этого листка нет листков с~ложными утверждениями.
\\
\emph {(2)} Ровно один листок левее этого содержит ложное утверждение.
\\
\emph {(3)} Ровно 2 листка левее этого содержат ложные утверждения.
\\\ldots\\
\emph{(11)} Ровно 10 листков левее этого содержат ложные утверждения.
\\
Листки в~некотором порядке выложили в~ряд, идущий слева направо.
После этого некоторые из~написанных утверждений стали верными, а~некоторые~---
неверными.
Каково наибольшее возможное число верных утверждений?

\end{problems}

