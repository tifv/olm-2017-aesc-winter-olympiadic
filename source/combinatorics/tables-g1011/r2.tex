% $date: 2017-01-08
% $timetable:
%   g1011r2:
%     2017-01-08:
%       3:

\section*{Таблички}

% $authors:
% - Антон Сергеевич Гусев

\begin{problems}

%\item
%В~клетки таблицы $m \times n$ вписаны некоторые числа.
%Разрешается одновременно менять знак у~всех чисел некоторого столбца или
%некоторой строки.
%Доказать, что многократным повторением этой операции можно превратить данную
%таблицу в~такую, у~которой суммы чисел, стоящих в~каждом столбце и~каждой
%строке, неотрицательны.

\item
Дана таблица $n \times n$, в~каждой клетке записано число, причем все числа
различны.
В~каждой строке отметили наименьшее число, и~все отмеченные числа оказались
в~разных столбцах.
Затем в~каждом столбце отметили наименьшее число, и~все отмеченные числа
оказались в~разных строках.
Докажите, что оба раза отметили одни и~те~же числа.

\item
\subproblem
Докажите, что если в~$3 n$ клетках таблицы $2 n \times 2 n$ расставлены $3 n$
звездочек, то~можно вычеркнуть $n$~столбцов и~$n$ строк так, что все звездочки
будут вычеркнуты.
\\
\subproblem
Докажите, что в~таблице $2 n \times 2 n$ можно расставить $3 n + 1$ звездочек
так, что при вычеркивании любых $n$~строк и~любых $n$~столбцов остается
невычеркнутой хотя~бы одна звездочка.

\item
В~таблицу $2 \times n$ (где $n > 2$~--- количество столбцов) вписаны числа.
Суммы во~всех столбцах различны.
Докажите, что можно переставить числа в~таблице так, чтобы суммы в~столбцах
были различны и~суммы в~строках были различны.

\item
В~некоторых клетках таблицы $10 \times 10$ расставлены несколько крестиков
и~несколько ноликов.
Известно, что нет линии (строки или столбца), полностью заполненной одинаковыми
значками (крестиками или ноликами).
Однако, если в~любую пустую клетку поставить любой значок, то~это условие
нарушится.
Какое минимальное число значков может стоять в~таблице?
\emph{(2009, 11.2)}

\item
Клетки шахматной доски $8 \times 8$ как-то занумерованы числами от~$1$ до~$32$,
причем каждое число использовано дважды.
Докажите, что можно так выбрать $32$~клетки, занумерованные разными числами,
что на~каждой вертикали и~на~каждой горизонтали найдется хотя~бы по~одной
выбранной клетке.

%\item
%На~небе бесконечное число звезд.
%Астроном приписал каждой звезде пару натуральных чисел, выражающую яркость
%и~размер.
%При этом каждые две звезды отличаются хотя~бы в~одном параметре.
%Докажите, что найдутся две звезды, первая из~которых не~меньше второй как
%по~яркости, так и~по~размеру.

\item
В~клетках доски $n \times n$ произвольно расставлены числа от~$1$ до~$n^2$.
Докажите, что найдутся две такие соседние клетки (имеющие общую вершину или
общую сторону), что стоящие в~них числа отличаются не~меньше чем на~$n + 1$.

%\item
%Все клетки квадратной таблицы $100 \times 100$ пронумерованы в~некотором
%порядке числами от~$1$ до~$10000$.
%Петя закрашивает клетки по~следующим правилам.
%Вначале он закрашивает $k$~клеток по~своему усмотрению.
%Далее каждым ходом Петя может закрасить одну еще не~закрашенную клетку
%с~номером~$a$, если для нее выполнено хотя~бы одно из~двух условий:
%\\
%\emph{(1)}
%в~одной строке с~ней есть уже закрашенная клетка с~номером меньшим, чем $a$;
%\\
%\emph{(2)}
%в~одном столбце с~ней есть уже закрашенная клетка с~номером большим, чем $a$.
%\\
%При каком наименьшем~$k$ независимо от~исходной нумерации Петя за~несколько
%ходов сможет закрасить все клетки таблицы?
%\emph{(2014, 9.4)}

\end{problems}

