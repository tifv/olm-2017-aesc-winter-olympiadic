% $date: 2017-01-10
% $timetable:
%   g9:
%     2017-01-10:
%       2:

\section*{Чередования}

% $authors:
% - Аскар Флоридович Назмутдинов

\begin{problems}

\item
Числа от~1 до~20 выписаны в~строчку.
Игроки по~очереди расставляют между ними плюсы и~минусы.
После того как все места заполнятся, подсчитывается результат.
Если он четен, то~выигрывает первый игрок, если нечетен, то~второй.
Кто выиграет при правильной игре?

\item
\begin{minipage}[t][][t]{0.78\linewidth}
В~таблице $4 \times 4$ расположены плюсы и~минусы, причем изначально в~ней
только один минус (рис.).
Разрешается менять все знаки на~противоположные в~любой строке, столбце, или
диагонали (на~диагонали любой длины, например можно менять угловую клетку).
Можно~ли получить в~ходе таких операций таблицу из~всех плюсов?
\end{minipage}%
\hfill
\begin{minipage}[t][][b]{0.20\linewidth}
    \vspace{-1ex}
    \jeolmfigure[width=\linewidth]{table}
\end{minipage}

\item
Можно~ли расставить по~кругу 1995 различных натуральных чисел так, чтобы для
любых двух соседних чисел отношение большего из~них к~меньшему было простым
числом?

\item
Семь лыжников с~номерами $1$, $2$,~\ldots, $7$ ушли со~старта по~очереди
и~прошли дистанцию~--- каждый со~своей постоянной скоростью.
Оказалось, что каждый лыжник ровно дважды участвовал в~обгонах.
(В~каждом обгоне участвуют ровно два лыжника~--- тот, кто обгоняет, и~тот, кого
обгоняют.)
По~окончании забега должен быть составлен протокол, состоящий из~номеров
лыжников в~порядке финиширования.
Докажите, что в~забеге с~описанными свойствами может получиться не~более двух
различных протоколов.

\item
Можно~ли в~таблице $11 \times 11$ расставить натуральные числа от~$1$ до~$121$
так, чтобы числа, отличающиеся друг от~друга на~единицу, располагались
в~клетках с~общей стороной, а~все точные квадраты попали в~один столбец?

\item
\begin{minipage}[t][][t]{0.65\linewidth}
На~рисунке показан треугольник, разбитый на~$25$ меньших треугольников,
занумерованных числами от~$1$ до~$25$.
Можно~ли эти~же числа расставить в~клетках квадрата $5 \times 5$ так, чтобы
любые два числа, записанные в~соседних треугольниках, были записаны
и~в~соседних клетках квадрата?
(Треугольники, так~же, как и~клетки квадрата, считаются соседними, если имеют
общую сторону.)
\end{minipage}%
\hfill
\begin{minipage}[t][][b]{0.33\linewidth}
    \vspace{-1ex}
    \jeolmfigure[width=\linewidth]{triangle}
\end{minipage}

\item
В~компании из~семи человек любые шесть могут сесть за~круглый стол так, что
каждые два соседа окажутся знакомыми.
Докажите, что и~всю компанию можно усадить за~круглый стол так, что каждые два
соседа окажутся знакомыми.

\end{problems}

