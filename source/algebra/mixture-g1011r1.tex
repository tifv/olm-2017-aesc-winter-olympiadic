% $date: 2017-01-10
% $timetable:
%   g1011r1:
%     2017-01-10:
%       3:

\section*{Алгебраический разнобой}

% $authors:
% - Юлий Васильевич Тихонов

\subsection*{Целочисленность}

\begin{problems}

\item
Бесконечная возрастающая арифметическая прогрессия такова, что произведение
любых двух ее членов~--- также член этой прогрессии.
Докажите, что все ее члены~--- целые числа.
\emph{(2012, 11.1)}

\item
Числа $x$, $y$ и $z$ таковы, что все три числа $x + y z$, $y + z x$ и $z + x y$
рациональны, а $x^2 + y^2 = 1$.
Докажите, что число $x y z^2$ также рационально.
\emph{(2014, 11.5)}

\item
Последовательность чисел $a_1$, $a_2$,~{\ldots} задана условиями
\[
    a_1 = 1
,\enspace
    a_2 = 143
,\enspace
    a_{n+1}
=
    5 \cdot \frac{a_{1} + a_{2} + \ldots + a_{n}}{n}
\]
при всех $n \geq 2$.
Докажите, что все члены последовательности~--- целые числа.
\emph{(2012, 10.3)}

\end{problems}

\subsection*{Неравенства}

\begin{problems}

\item
Числа $a$ и $b$ таковы, что $a^3 - b^3 = 2$,
$a^5 - b^5 > 4$.
Докажите, что $a^2 + b^2 > 2$.
\emph{(2012, 9.6)}

\item
Положительные числа $x$, $y$ и~$z$ удовлетворяют условию $xyz > xy + yz + zx$.
Докажите неравенство
$\sqrt{xyz} > \sqrt{x} + \sqrt{y} + \sqrt{z}$.
\emph{(2016, 11.2)}

\item
Положительные числа $a$, $b$, $c$ удовлетворяют соотношению
$ab + bc + ca = 1$.
Докажите, что
\[
    \sqrt{a + \frac{1}{a}} +
    \sqrt{b + \frac{1}{b}} +
    \sqrt{c + \frac{1}{c}}
\geq
    2 (\sqrt{a} + \sqrt{b} + \sqrt{c})
\, . \]
\emph{(2016, 10.4, 11.4)}

\end{problems}

\subsection*{11 класс}

\begin{problems}

\item
Даны различные натуральные числа $a$, $b$.
На координатной плоскости нарисованы графики функций
$y = \sin(ax)$ и $y = \sin(bx)$ и отмечены все точки их пересечения.
Докажите, что существует натуральное число $c$, отличное от $a$, $b$ и~такое,
что график функции $y = \sin(c x)$ проходит через все отмеченные точки.
\emph{(2012, 11.7)}

\end{problems}

%\item
%Натуральное число $N$ представляется в~виде
%$N = a_1 - a_2 = b_1 - b_2 = c_1 - c_2 = d_1 - d_2$,
%где $a_1$ и~$a_2$~--- квадраты, $b_1$ и~$b_2$~--- кубы,
%$c_1$ и~$c_2$~--- пятые степени, а~$d_1$ и~$d_2$~--- седьмые степени
%натуральных чисел.
%Обязательно~ли среди чисел $a_1$, $b_1$, $c_1$ и~$d_1$ найдутся два равных?
%\emph{(2016, 11.8)}

%\item
%По~кругу расставлено 300 положительных чисел.
%Могло~ли случиться так, что каждое из~этих чисел, кроме одного, равно разности
%своих соседей?
%\emph{(2015, 11.7)}

