% $date: 2017-01-10
% $timetable:
%   g9:
%     2017-01-10:
%       3:

\section*{Неравенство Коши}

% $authors:
% - Виктория Владимировна Журавлёва

\begin{problems}

\item
Докажите, что $(x + y) / 2 \geq \sqrt{x y}$ при любых $x \geq 0$ и~$y \geq 0$.

\item
Докажите, что при $a, b, c > 0$ имеет место неравенство
\[
    \left(
        \frac{a + b + c}{3}
    \right)^2
\geq
    \frac{a b + b c + c a}{3}
\, . \]

\item
Произведение двух положительных чисел больше их суммы.
Докажите, что эта сумма больше четырех.

\item
Найдите все неотрицательные решения системы уравнений:
\[
\left\{\begin{aligned} &
    x^3 = 2 y^2 - z
, \\ &
    y^3 = 2 z^2 - x
, \\ &
    z^3 = 2 x^2 - y
. \end{aligned}\right.
\]

\item
Числа $a$, $b$, $c$ и~$d$ таковы, что $a^2 + b^2 + c^2 + d^2 = 4$.
Докажите, что $(2 + a) (2 + b) \geq c d$.

\item
Даны различные натуральные числа $a_{1}, a_{2}, \ldots, a_{14}$.
На~доску выписаны все $196$ чисел вида $a_{k} + a_{l}$, где
$1 \leq k, l \leq 14$.
Может~ли оказаться, что для любой комбинации из~двух цифр среди написанных
на~доске чисел найдется хотя~бы одно число, оканчивающееся на~эту комбинацию
(то~есть, найдутся числа, оканчивающиеся на~$00, 01, 02, \ldots, 99$)?

\end{problems}

