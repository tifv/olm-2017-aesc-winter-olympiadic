% $date: 2017-01-11
% $timetable:
%   g1011r1:
%     2017-01-11:
%       2:

\section*{Квадратный трехчлен}

% $authors:
% - Юлий Васильевич Тихонов

\begin{problems}

%\item
%Квадратный трехчлен $f(x) = a x^2 + b x + c$, не~имеющий корней, таков, что
%коэффициент $b$ рационален, а~среди чисел $c$ и~$f(c)$ ровно одно
%иррационально.
%Может~ли дискриминант трехчлена $f(x)$ быть рациональным?
%\emph{(2016, 11.1)}

%\item
%Квадратный трехчлен $f(x)$ имеет два различных корня.
%Оказалось, что для любых чисел $a$ и~$b$ верно неравенство
%$f(a^2 + b^2) > f(2 a b)$.
%Докажите, что хотя~бы один из~корней этого трехчлена~--- отрицательный.
%\emph{(2015, 11.5)}

%\item
%Ненулевые числа $a$, $b$, $c$ таковы, что $a x^2 + b x + c > c x$ при
%любом $x$.
%Докажите, что $c x^2 - b x + a > c x - b$ при любом $x$.
%\emph{(2010, 10.5)}

%\item
%На~доске написано уравнение $x^3 + \ast x^2 + \ast x + \ast = 0$.
%Петя и~Вася по~очереди заменяют звездочки на~рациональные числа:
%вначале Петя заменяет любую из~звездочек, потом Вася~--- любую из~двух
%оставшихся, а~затем Петя~--- оставшуюся звездочку.
%Верно~ли, что при любых действиях Васи Петя сможет получить уравнение,
%у~которого разность каких-то двух корней равна 2014?
%\emph{(2014, 10.5)}

\item
Даны различные действительные числа $a$, $b$, $c$.
Докажите, что хотя~бы два из~уравнений
$(x - a) (x - b) = x - c$, $(x - b) (x - c) = x - a$, $(x - c) (x - a) = x - b$
имеют решение.

%\item
%$P(x)$ и~$Q(x)$~--- приведенные квадратные трехчлены, имеющие по~два различных
%корня.
%Оказалось, что сумма двух чисел, получаемых при подстановке корней
%трехчлена $P(x)$ в~трехчлен $Q(x)$, равна сумме двух чисел, получаемых при
%подстановке корней трехчлена $Q(x)$ в~трехчлен $P(x)$.
%Докажите, что дискриминанты трехчленов $P(x)$ и~$Q(x)$ равны.
%\emph{(2013, 11.2)}

%\item
%Целые числа $a$, $b$, $c$ таковы, что значения квадратных трехчленов
%$b x^2 + c x + a$ и~$c x^2 + a x + b$ при $x = 1234$ совпадают.
%Может~ли первый трехчлен при $x = 1$ принимать значение $2009$?
%\emph{(2010, 11.7)}

\item
Даны три квадратных трехчлена $P(x)$, $Q(x)$ и~$R(x)$ с~положительными старшими
коэффициентами, имеющие по~два различных корня.
Оказалось, что при подстановке корней трехчлена~$R(x)$ в~многочлен
$P(x) + Q(x)$ получаются равные значения.
Аналогично, при подстановке корней трехчлена $P(x)$ в~многочлен $Q(x) + R(x)$
получаются равные значения, а~также при подстановке корней трехчлена $Q(x)$
в~многочлен $P(x) + R(x)$ получаются равные значения.
Докажите, что три числа:
сумма корней трехчлена $P(x)$, сумма корней трехчлена $Q(x)$ и~сумма
корней трехчлена $R(x)$ равны между собой.
\emph{(2013, 10.3)}

\end{problems}

\subsection*{Почти}

\begin{problems}

\item
Дан многочлен
\[
    P(x) = a_{2n} x^{2n} + a_{2n-1} x^{2n-1} + \ldots + a_{1} x + a_{0}
\, , \]
у~которого каждый коэффициент $a_{i}$ принадлежит отрезку $[100; 101]$.
При каком минимальном $n$ у~такого многочлена может найтись действительный
корень?
\emph{(2014, 11.7)}

\end{problems}

