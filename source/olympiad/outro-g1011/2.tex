% $date: 2017-01-12
% $timetable:
%   g1011r2:
%     2017-01-12: {}
%   g1011r1:
%     2017-01-12: {}

% $matter[no-header,-contained]:
% - verbatim: \section*{Заключительная олимпиада}
% - .[contained,setproblem]

% $matter[no-header,contained,add-toc]:
% - .[-add-toc]

% $matter[no-header,contained,-add-toc,setproblem]:
% - verbatim: \setproblem{5}
% - .[-setproblem]

% $groups$delegate: -g1011
% $groups$matter: -g1011
% $matter[g1011,-no-header,-groups-guard]:
% - verbatim: \begingroup \def\jeolmgroupname{10--11 класс}
% - .[groups-guard]
% - verbatim: \endgroup

\subsection*{Вывод}

\begin{problems}

\item
Дано натуральное число $n > 6$.
Рассматриваются натуральные числа, лежащие в~промежутке
$\bigl( n (n - 1), n^2 \bigr)$ и~взаимно простые с~$n (n - 1)$.
Докажите, что наибольший общий делитель всех таких чисел равен 1.

\item
Точки $A_2$, $B_2$ и~$C_2$~--- середины высот $A A_1$, $B B_1$ и~$C C_1$
остроугольного треугольника $ABC$.
Найдите сумму углов $B_2 A_1 C_2$, $C_2 B_1 A_2$ и~$A_2 C_1 B_2$.

\item
В~каждой клетке квадратной таблицы $m \times m$ клеток стоит либо натуральное
число, либо нуль.
При этом, если на~пересечении строки и~столбца стоит нуль, то~сумма чисел
в~<<кресте>>, состоящем из~этой строки и~этого столбца, не~меньше $m$.
Докажите, что сумма всех чисел в~таблице не~меньше, чем $m^2 / 2$.

\end{problems}

