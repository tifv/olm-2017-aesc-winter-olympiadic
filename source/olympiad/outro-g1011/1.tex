% $date: 2017-01-12
% $timetable:
%   g1011r2:
%     2017-01-12: {}
%   g1011r1:
%     2017-01-12: {}

\section*{Заключительная олимпиада}

% $groups$delegate: -g1011
% $groups$matter: -g1011
% $matter[g1011,-groups-guard]:
% - verbatim: \begingroup \def\jeolmgroupname{10--11 класс}
% - .[groups-guard]
% - verbatim: \endgroup

\subsection*{Довывод}

\begingroup
    \def\abs#1{\lvert #1 \rvert}%

\begin{problems}

\item
Какое наибольшее число белых и~черных фишек можно расставить на~шахматной доске
так, чтобы на~каждой горизонтали и~на~каждой вертикали белых фишек было ровно
в~два раза больше, чем черных?

\item
Взяли три числа $x$, $y$, $z$.
Вычислили абсолютные величины попарных разностей
$x_{1} = \abs{x - y}$, $y_{1} = \abs{y - z}$, $z_{1} = \abs{z - x}$.
Тем~же способом по~числам $x_{1}$, $y_{1}$, $z_{1}$ построили числа $x_{2}$,
$y_{2}$, $z_{2}$ и~т.\,д.
При некотором $n$ оказалось, что $x_{n} = x$, $y_{n} = y$, $z_{n} = z$.
Зная, что $x = 1$, найдите $y$ и~$z$.

\item
Окружности $S_1$ и~$S_2$ с~центрам $O_1$ и~$O_2$ соответственно пересекаются
в~точках $A$ и~$B$.
Касательные к~$S_1$ и~$S_2$ в~точке~$A$ пересекают отрезки $B O_2$ и~$B O_1$
в~точках $K$ и~$L$ соответственно.
Докажите, что $KL \parallel O_1 O_2$.

\item
Есть 100 коробок, пронумерованных числами от~1 до~100.
В~одной коробке лежит приз и~ведущий знает, где он находится.
Зритель может один раз послать ведущему список вопросов, требующих ответа
<<да>> или <<нет>>.
Ведущий в~каком-то порядке (зритель не~знает, в~каком именно) честно отвечает
на~все вопросы.
Какое наименьшее количество вопросов нужно написать зрителю, чтобы наверняка
узнать, где находится приз?

\item
Целые числа $a$, $b$, $c$ таковы, что значения квадратных трехчленов
$b x^2 + c x + a$ и~$c x^2 + a x + b$ при $x = 1234$ совпадают.
Может~ли первый трехчлен при $x = 1$ принимать значение $2017$?

\end{problems}

\endgroup % \def\abs

