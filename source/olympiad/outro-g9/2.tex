% $date: 2017-01-12
% $timetable:
%   g9:
%     2017-01-12: {}

% $matter[no-header,-contained]:
% - verbatim: \section*{Заключительная олимпиада}
% - .[contained,setproblem]

% $matter[no-header,contained,add-toc]:
% - .[-add-toc]

% $matter[no-header,contained,-add-toc,setproblem]:
% - verbatim: \setproblem{5}
% - .[-setproblem]

\subsection*{Вывод}

\begin{problems}

\item
Найдите наибольшее натуральное число, из~которого вычеркиванием цифр нельзя
получить число, кратное 11.

\item
Треугольник $ABC$ вписан в~окружность~$S$.
Пусть $A_0$~--- середина дуги~$BC$ окружности~$S$, не~содержащей $A$;
$C_0$~--- середина дуги~$AB$, не~содержащей $C$.
Окружность~$S_1$ с~центром~$A_0$ касается $BC$,
окружность~$S_2$ с~центром~$C_0$ касается $AB$.
Докажите, что центр~$I$ вписанной в~треугольник $ABC$ окружности лежит на~одной
из~общих внешних касательных к~окружностям $S_1$ и~$S_2$.

\item
На~прямой стоят две фишки, слева~--- красная, справа~--- синяя.
Разрешается производить любую из~двух операций:
вставку двух фишек одного цвета подряд в~любом месте прямой
и~удаление любых двух соседних одноцветных фишек
(фишки \emph{соседние,} если между ними нет других фишек).
Можно~ли за~конечное число операций оставить на~прямой ровно две фишки:
красную справа, а~синюю~--- слева?

\end{problems}

