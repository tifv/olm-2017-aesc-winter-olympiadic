% $date: 2017-01-08
% $timetable:
%   g9:
%     2017-01-08: {}

% $matter[no-header,-contained]:
% - verbatim: \section*{Вступительная олимпиада}
% - .[contained,setproblem]

% $matter[no-header,contained,add-toc]:
% - .[-add-toc]

% $matter[no-header,contained,-add-toc,setproblem]:
% - verbatim: \setproblem{5}
% - .[-setproblem]

\begingroup
    \providecommand\ifsolutions{\iffalse}
    \long\def\solution#1{\ifsolutions\emph{Решение.} {#1}\fi}
    \def\abs#1{\lvert #1 \rvert}

\subsection*{Вывод}

\begin{problems}

\item
Окружности с~центрами $O_1$ и~$O_2$ пересекаются в~точках $A$ и~$B$.
Описанная окружность треугольника $O_1 B O_2$ пересекает окружности
в~точках $C$ и~$D$, а~прямую~$AB$~--- в~точке~$X$.
Докажите, что $X$ является центром описанной окружности треугольника $CAD$.

\item
Ко~дню Российского Флага продавец украсил витрину $12$ горизонтальными
полосками ткани трех цветов.
При этом он выполняет два условия:
\\
\emph{(1)}\enspace
одноцветные полосы не~должны висеть рядом;
\\
\emph{(2)}\enspace
каждая синяя полоса должна висеть между белой и~красной.
\\
Сколькими способами он может это сделать?

  \solution{%
    Пусть $R_{n}$~--- количество способов расставить $n$ полосок по~указанным
    правилам, если первая полоска белая.
    Заметим, что $R_{1} = R_{2} = 1$.
    Также заметим, что $R_{n+2} = R_{n+1} + R_{n}$.
    Действительно, из~$n + 2$ полосок предпоследняя либо синяя, либо нет.
    В~первом случае первые $n$ полосок образуют одну из~$R_{n}$ расстановок,
    соответствующих правилам, а~последние две определяются однозначно.
    Во~втором случае первые $n + 1$ полос образуют одну из~$R_{n+1}$
    расстановок, а~оставшаяся определяется однозначно.
    Отсюда следует, что $R_{n}$~--- это $n$-е число Фибоначчи.
    Ответ в~два раза больше, так как расстановок, начинающихся с~красной
    полоски, еще столько~же.
    Это $288$.}

\item
Известно, что $0 < a, b, c, d < 1$
и~$a b c d = (1 - a) (1 - b) (1 - c) (1 - d)$.
Докажите, что $(a + b + c + d) - (a + c) (b + d) \geq 1$.

  \solution{%
    Искомое неравенство нетрудно переписать в~виде
    $(1 - a - c) (1 - b - d) \leq 0$.
    Таким образом, нужно доказать, что числа $(1 - a - c)$ и~$(1 - b - d)$
    находятся по~разные стороны от~$0$ (или одно из~них равно $0$).

    С~другой стороны, перепишем соотношение, данное по~условию, как
    \[
        \frac{(1 - a) (1 - c)}{a c}
        \cdot
        \frac{(1 - b) (1 - d)}{b d}
    =
        1
    \, . \]
    Раскрыв скобки в~числителях и~отделив $1$ от~каждой дроби, получаем
    \[
        \left( \frac{1 - a - c}{a c} + 1 \right)
        \cdot
        \left( \frac{1 - b - d}{b d} + 1 \right)
    =
        1
    \, . \]
    Очевидно, сомножители взаимно обратны и~находятся по~разные стороны от~$1$.
    Но~тогда $(1 - a - c)$ и~$(1 - b - d)$ находятся по~разные стороны от~$0$.}

\end{problems}

\endgroup

