% $date: 2017-01-08
% $timetable:
%   g9:
%     2017-01-08: {}

\section*{Вступительная олимпиада}

\begingroup
    \providecommand\ifsolutions{\iffalse}
    \long\def\solution#1{\ifsolutions\emph{Решение.} {#1}\fi}
    \def\abs#1{\lvert #1 \rvert}

\subsection*{Довывод}

\begin{problems}

\item
На доске написаны несколько чисел.
Известно, что квадрат любого записанного числа больше произведения любых двух
других записанных чисел.
Какое наибольшее количество чисел может быть на доске?

\item
Какую минимальную сумму цифр может иметь натуральное число, делящееся на~$99$?
% http://problems.ru/view_problem_details_new.php?id=35790

  \solution{%
     Понятно, что сумма цифр должна делится на~$9$.
     Но~если она равна $9$, то~признак делимости на~$11$ не~может быть выполнен
     (если сумма цифр на~четных позициях $a$, а~на~нечетных $b$,
     то~$a + b = 9$, откуда $\abs{a - b} < 11$, то~есть $a - b = 0$ из~признака
     делимости на~$11$, но~тогда $a = b = 4{,}5$).
     Значит, минимальная сумма цифр~--- $18$.}

\item
Одна из~сторон вписанного четырехугольника является диаметром окружности.
Докажите, что проекции сторон, прилегающих к~этой стороне, на~прямую, задающую
четвертую сторону, равны между собой.
% http://problems.ru/view_problem_details_new.php?id=53617

  \solution{%
    Обозначим четырехугольник $ABCD$, где $AB$~--- диаметр окружности;
    а~центр окружности пусть будет $O$.
    Проекции на~$CD$ обозначим соответственно $A'$, $B'$, $O'$.
    Тогда $O'$~--- середина $CD$ (центр окружности проецируется в~середину
    хорды), и~$O'$~--- середина $A'B'$ (середина отрезка проецируется
    в~середину).
    Отсюда легко следует, что $A'D = B'C$.}

\item
Через центры некоторых клеток шахматной доски $8 \times 8$ проведена замкнутая
несамопересекающаяся ломаная.
Каждое звено ломаной соединяет центры соседних по~горизонтали, вертикали или
диагонали клеток.
Докажите, что в~ограниченном ею многоугольнике общая площадь черных частей
равна общей площади белых частей.
% http://www.problems.ru/view_problem_details_new.php?id=116542

  \solution{%
    Проведем всевозможные отрезки, которые могут быть звеньями нашей ломаной
    согласно условию.
    Заметим, что они образовали квадрат, разбитый на~маленькие треугольнички.
    Очевидно, что каждый такой треугольничек либо целиком окажется внутри
    ломаной, либо целиком снаружи;
    следовательно, наш многоугольник разобьется на~такие треугольнички.
    Осталось заметить, что в~каждом треугольничке белого и~черного поровну.}

\item
Двое по~очереди выписывают на~доску натуральные числа от~$1$ до~$1000$.
Первым ходом первый игрок выписывает на~доску число~$1$.
Затем очередным ходом на~доску можно выписать либо число~$2 a$, либо число
$a + 1$, если на~доске уже написано число~$a$.
При этом запрещается выписывать числа, которые уже написаны на~доске.
Выигрывает тот, кто выпишет на~доску число $1000$.
Кто выигрывает при правильной игре?
% http://problems.ru/view_problem_details_new.php?id=110141

  \solution{%
    Легко заметить, что игрок, выписавший любое из~чисел $500$ и~$999$,
    немедленно проиграет.
    Пусть $A$~--- множество всех чисел от~$1$ до~$1000$, кроме $1000$, $500$,
    $999$ и $501$.
    Тогда у~первого игрока появляется <<оттягивающая>> стратегия:
    если второй игрок выписал одно из~чисел $500$ или $999$, то~первый
    выписывает $1000$;
    в~противном случае он выписывает любое (например, наименьшее) еще
    не~выписанное число из~$A$.
    Он может это сделать, так как к~его ходу только четное количество чисел
    из~$A$ будет выписано, а~всего их там нечетно; число $501$ при этом
    не~будет выписано никогда.}

\end{problems}

\endgroup

