% $date: 2017-01-08
% $timetable:
%   g1011r2:
%     2017-01-08: {}
%   g1011r1:
%     2017-01-08: {}

% $matter[no-header,-contained]:
% - verbatim: \section*{Вступительная олимпиада}
% - .[contained,setproblem]

% $matter[no-header,contained,add-toc]:
% - .[-add-toc]

% $matter[no-header,contained,-add-toc,setproblem]:
% - verbatim: \setproblem{5}
% - .[-setproblem]

% $groups$delegate: -g1011
% $groups$matter: -g1011
% $matter[g1011,-no-header,-groups-guard]:
% - verbatim: \begingroup \def\jeolmgroupname{10--11 класс}
% - .[groups-guard]
% - verbatim: \endgroup

\subsection*{Вывод}

\begin{problems}

\item
Существуют~ли три попарно различных ненулевых целых числа, сумма которых равна
нулю, а~сумма тринадцатых степеней которых является квадратом некоторого
натурального числа?

\item
Назовем \emph{лестницей высоты $n$} фигуру, состоящую из~всех клеток
квадрата $n \times n$, лежащих не~выше диагонали.
Сколькими различными способами можно разбить лестницу высоты~$n$ на~несколько
прямоугольников, стороны которых идут по~линиям сетки, а~площади попарно
различны?

\item
Точка~$O$ лежит внутри ромба $ABCD$.
Угол $DAB$ равен $110^{\circ}$.
Углы $AOD$ и~$BOC$ равны $80^{\circ}$ и~$100^{\circ}$ соответственно.
Чему может быть равен угол $AOB$?
% решение: 9.5, 2008 год отсюда
% http://www.mccme.ru/free-books/olymp/mmo1993.pdf

%\item
%Назовем \emph{лестницей высоты~$n$} фигуру, состоящую из~всех клеток квадрата
%$n \times n$, лежащих не~выше диагонали.
%Сколькими различными способами можно разбить лестницу высоты~$n$ на~несколько
%прямоугольников, стороны которых идут по~линиям сетки, а~площади попарно
%различны?

\end{problems}

