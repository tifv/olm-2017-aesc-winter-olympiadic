% $date: 2017-01-08
% $timetable:
%   g1011r2:
%     2017-01-08: {}
%   g1011r1:
%     2017-01-08: {}

\section*{Вступительная олимпиада}

% $groups$delegate: -g1011
% $groups$matter: -g1011
% $matter[g1011,-groups-guard]:
% - verbatim: \begingroup \def\jeolmgroupname{10--11 класс}
% - .[groups-guard]
% - verbatim: \endgroup

\subsection*{Довывод}

\begin{problems}

\item
Найдите все $x$, при которых $\tg x$ и~$\tg 2x$ одновременно принимают целые
значения.
% пусть tg x = a
% тогда tg 2x = 2a / (1 - a^2) = 2a / (1 - a)(1 + a)
% так как a взаимно просто с 1 - a и 1 + a, то 1 - a^2 = \pm 1, \pm 2
% такие дела

\item
Значения квадратного трехчлена $y = x^2 + ax + b$ в~двух последовательных целых
точках~--- соответственно квадраты двух последовательных натуральных чисел.
Докажите, что значения трехчлена во~всех целых точках~--- точные квадраты.
% пусть в точке x имеем n^2, в точке x + 1 имеем (n + 1)^2
% Тогда
%   (n + 1)^2 - n^2 = 2n + 1 =
%   (x + 1)^2 + a(x + 1) + b - x^2 - ax - b = 2x + 1 + a
% значит n = x + a/2, в частности, a четно и равно 2c
% из (x + c)^2 = x^2 + 2cx + b имеем b = c^2, то есть у нас полный квадрат

\item
В~волейбольном турнире с~участием $73$~команд каждая команда сыграла с~каждой
по~одному разу.
В~конце турнира все команды разделили на~две непустые группы так, что
каждая команда первой группы одержала ровно $n$~побед,
а~каждая команда второй группы~--- ровно $m$~побед.
Могло~ли оказаться, что $m \neq n$?
% это задача 11.6 отсюда http://olympiads.mccme.ru/vmo/2012/iii-2.pdf

\item
Треугольник $ABC$ вписан в~окружность~$\Omega$ с~центром~$O$.
Окружность, построенная на~$AO$ как на~диаметре, пересекает описанную
окружность треугольника $OBC$ в~точке $S \neq O$.
Касательные к~$\Omega$ в~точках $B$ и~$C$ пересекаются в~точке~$P$.
Докажите, что точки $A$, $S$ и~$P$ лежат на~одной прямой.

\item
Пусть $p > 5$~--- простое число.
Из~десятичной записи числа $1 / p$ выкинули $2017$-ю цифру после запятой
и~получили число, представимое в~виде несократимой дроби $a / b$.
Докажите, что $b$ делится на~$p$.
% можно записать два уравнения
% 1/p = x / 10^k + y / 10^{k + 1} + t
% a/b = x / 10^k + 10t
% умножаем первое на 10 и вычитаем второе
% 10/p - a / b = (10b - ap) / bp = 9x + y/ 10^k
% так как p взаимно просто с 10, средняя дробь сократима на p, то есть 10b
% кратно p, а значит и b кратно p

\end{problems}

