% $date: 2017-01-09
% $timetable:
%   g1011r2:
%     2017-01-09:
%       2:

\section*{Вписанные углы --- 1}

% $authors:
% - Алексей Вадимович Доледенок

\begin{problems}

\item
Из~точки~$A$, лежащей внутри угла с~вершиной~$O$, опущены перпендикуляры $AB$
и~$AC$ на~стороны угла.
Из~точки~$O$ опущен перпендикуляр~$OP$ на~отрезок~$BC$.
Докажите, что $\angle POC = \angle AOB$.

\item
Дан равнобедренный треугольник $ABC$ ($AB = AC$).
На~меньшей дуге~$AB$ описанной около него окружности взята точка~$D$.
На~продолжении отрезка~$AD$ за~точку~$D$ выбрана точка~$E$ так,
что точки $A$ и~$E$ лежат в~одной полуплоскости относительно $BC$.
Описанная окружность треугольника $BDE$ пересекает сторону~$AB$ в~точке~$F$.
Докажите, что прямые $EF$ и~$BC$ параллельны.

\item
Внутри равнобокой трапеции $ABCD$ с~основаниями $BC$ и~$AD$ расположена
окружность~$\omega$ с~центром~$I$, касающаяся отрезков $AB$, $CD$ и~$DA$.
Окружность, описанная около треугольника $BIC$, вторично пересекает
сторону~$AB$ в~точке~$E$.
Докажите, что прямая~$CE$ касается окружности~$\omega$.

\item
Пусть точки $A$, $B$, $C$ лежат на~окружности, а~прямая~$b$ касается этой
окружности в~точке~$B$.
Из~точки~$P$, лежащей на~прямой~$b$, опущены перпендикуляры $P A_1$ и~$P C_1$
на~прямые $AB$ и~$BC$ соответственно
(точки $A_1$ и~$C_1$ лежат на~отрезках $AB$ и~$BC$).
Докажите, что $A_1 C_1 \perp AC$.

\item
На~стороне~$AC$ остроугольного треугольника $ABC$ выбраны точки $M$ и~$K$ так,
что $\angle ABM = \angle CBK$.
Докажите, что центры окружностей, описанных около треугольников $ABM$, $ABK$,
$CBM$ и~$CBK$, лежат на~одной окружности.

\item
В~остроугольном треугольнике $ABC$ провели высоты $A A_1$ и~$C C_1$.
Окружность~$\Omega$, описанная около треугольника $ABC$, пересекает
прямую~$A_1 C_1$ в~точках $A'$ и~$C'$.
Касательные к~$\Omega$, проведенные в~точках $A'$ и~$C'$, пересекаются
в~точке~$B'$.
Докажите, что прямая~$BB'$ проходит через центр окружности~$\Omega$.

\item
Вписанная в~треугольник $ABC$ окружность~$\omega$ касается
сторон $BC$, $CA$, $AB$ в~точках $A_1$, $B_1$ и~$C_1$ соответственно.
На~продолжении отрезка~$A A_1$ за~точку~$A$ взята точка~$D$ такая, что
$AD = A C_1$.
Прямые $D B_1$ и~$D C_1$ пересекают второй раз окружность~$\omega$
в~точках $B_2$ и~$C_2$.
Докажите, что $B_2 C_2$~--- диаметр окружности~$\omega$.

\item
Дан выпуклый шестиугольник $ABCDEF$.
Известно, что $\angle FAE = \angle BDC$, а~четырехугольники $ABDF$ и~$ACDE$
являются вписанными.
Докажите, что прямые $BF$ и~$CE$ параллельны.

\item
Окружности с~центрами $O_1$ и~$O_2$ пересекаются в~точках $A$ и~$B$.
Описанная окружность треугольника $O_1 B O_2$ пересекает окружности
в~точках $C$ и~$D$, а~прямую~$AB$~--- в~точке~$X$.
Докажите, что $X$ является центром описанной окружности треугольника $CAD$.

\end{problems}

