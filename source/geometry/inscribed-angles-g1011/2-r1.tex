% $date: 2017-01-09
% $timetable:
%   g1011r1:
%     2017-01-09:
%       1:

\section*{Вписанные углы --- 2}

% $authors:
% - Алексей Вадимович Доледенок

\begin{problems}

\item
Внутри равнобокой трапеции $ABCD$ с~основаниями $BC$ и~$AD$ расположена
окружность~$\omega$ с~центром~$I$, касающаяся отрезков $AB$, $CD$ и~$DA$.
Окружность, описанная около треугольника $BIC$, вторично пересекает
сторону~$AB$ в~точке~$E$.
Докажите, что прямая~$CE$ касается окружности~$\omega$.

\item
Пусть точки $A$, $B$, $C$ лежат на~окружности, а~прямая~$b$ касается этой
окружности в~точке~$B$.
Из~точки~$P$, лежащей на~прямой~$b$, опущены перпендикуляры $P A_1$ и~$P C_1$
на~прямые $AB$ и~$BC$ соответственно
(точки $A_1$ и~$C_1$ лежат на~отрезках $AB$ и~$BC$).
Докажите, что $A_1 C_1 \perp AC$.

\item
Дан выпуклый шестиугольник $ABCDEF$.
Известно, что $\angle FAE = \angle BDC$, а~четырехугольники $ABDF$ и~$ACDE$
являются вписанными.
Докажите, что прямые $BF$ и~$CE$ параллельны.

\item
Дан параллелограмм $ABCD$, в~котором угол $ABC$ тупой.
Прямая~$AD$ пересекает второй раз окружность~$\omega$, описанную вокруг
треугольника $ABC$, в~точке~$E$.
Прямая~$CD$ пересекает второй раз окружность~$\omega$ в~точке~$F$.
Докажите, что центр описанной окружности треугольника $DEF$ лежит
на~окружности~$\omega$.

\item
В~треугольнике $ABC$ проведены биссектрисы $AD$, $BE$ и~$CF$, пересекающиеся
в~точке~$I$.
Серединный перпендикуляр к~отрезку~$AD$ пересекает прямые $BE$ и~$CF$
в~точках $M$ и~$N$ соответственно.
Докажите, что точки $A$, $I$, $M$ и~$N$ лежат на~одной окружности.

\item
В~выпуклом четырехугольнике $ABCD$ провели биссектрисы
$l_{a}$, $l_{b}$, $l_{c}$, $l_{d}$ внешних углов $A$, $B$, $C$, $D$
соответственно.
Точки пересечения прямых $l_{a}$ и~$l_{b}$, $l_{b}$ и~$l_{c}$,
$l_{c}$ и~$l_{d}$, $l_{d}$ и~$l_{a}$ обозначили через $K$, $L$, $M$, $N$.
Известно, что три перпендикуляра, опущенных из~$K$ на~$AB$, из~$L$ на~$BC$,
из~$M$ на~$CD$, пересекаются в~одной точке.
Докажите, что четырехугольник $ABCD$ вписанный.

\item
Дан выпуклый четырехугольник $ABCD$.
Описанная окружность треугольника $ABC$ пересекает стороны $AD$ и~$DC$
в~точках $P$ и~$Q$ соответственно.
Описанная окружность треугольника $ADC$ пересекает стороны $AB$ и~$BC$
в~точках $S$ и~$R$ соответственно.
Оказалось, что четырехугольник $PQRS$~--- параллелограмм.
Докажите, что $ABCD$~--- также параллелограмм.

\item
В~остроугольном треугольнике $ABC$ точка~$M$~--- cередина стороны~$AC$,
$AD$~--- высота.
Внутри треугольника $ABC$ отметили точку~$X$ такую, что
$\angle AXB = \angle DXM = 90^{\circ}$.
Докажите, что $\angle XMB = 2 \angle MBC$.

\end{problems}

