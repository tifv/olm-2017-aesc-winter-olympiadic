% $date: 2017-01-10
% $timetable:
%   g1011r2:
%     2017-01-10:
%       1:

\section*{Вписанные углы --- 2}

% $authors:
% - Алексей Вадимович Доледенок

\begin{problems}

\item
Точка~$F$~--- середина стороны~$BC$ квадрата $ABCD$.
К~отрезку~$DF$ проведен перпендикуляр~$AE$.
Найдите угол $CEF$.

\item
Четырехугольник $ABCD$~--- вписанный.
На~его диагоналях $AC$ и~$BD$ отметили точки $K$ и~$L$ соответственно так, что
$AK = AB$ и~$DL = DC$.
Докажите, что прямые $KL$ и~$AD$ параллельны.

\item
Через вершины $B$ и~$C$ основания~$BC$ трапеции $ABCD$ провели окружность,
которая пересекла боковые стороны $AB$ и~$CD$ в~точках $M$ и~$N$
соответственно.
Известно, что точка~$T$ пересечения отрезков $AN$ и~$DM$ также лежит на~этой
окружности.
Докажите, что $TB = TC$.

\item
Дана равнобокая трапеция $ABCD$ ($AD \parallel BC$).
На~дуге~$AD$ (не~содержащей точек $B$ и~$C$) описанной окружности этой трапеции
произвольно выбрана точка~$M$.
Докажите, что основания перпендикуляров, опущенных из~вершин $A$ и~$D$
на~отрезки $BM$ и~$CM$, лежат на~одной окружности.

\item
Дан параллелограмм $ABCD$, в~котором угол $ABC$ тупой.
Прямая~$AD$ пересекает второй раз окружность~$\omega$, описанную вокруг
треугольника $ABC$, в~точке~$E$.
Прямая~$CD$ пересекает второй раз окружность~$\omega$ в~точке~$F$.
Докажите, что центр описанной окружности треугольника $DEF$ лежит
на~окружности~$\omega$.

\item
Дан остроугольный треугольник $ABC$.
Окружности с~центрами $A$ и~$C$ проходят через точку~$B$, вторично пересекаются
в~точке~$F$ и~пересекают описанную окружность~$\omega$ треугольника $ABC$
в~точках $D$ и~$E$.
Отрезок~$BF$ пересекает окружность~$\omega$ в~точке~$O$.
Докажите, что $O$~--- центр описанной окружности треугольника $DEF$.

\end{problems}

