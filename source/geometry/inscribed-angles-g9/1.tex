% $date: 2017-01-09
% $timetable:
%   g9:
%     2017-01-09:
%       1:

\section*{Вписанные углы}

% $authors:
% # Алексей Александрович Пономарёв
% - Юлий Васильевич Тихонов

\begin{problems}
    \let\bfseries\relax

\item
Высоты $A A_1$, $B B_1$ и~$C C_1$ треугольника $ABC$ пересекаются в~точке~$H$.
Докажите, что
\\
\subproblem четырехугольник $A B_1 H C_1$ вписанный;
\\
\subproblem $\angle BAH = \angle H B_1 A_1$;
\\
\subproblem $A A_1$~--- биссектриса треугольника $A_1 B_1 C_1$.

\item
\subproblem
Вершина~$A$ остроугольного треугольника $ABC$ соединена отрезком с~центром~$O$
описанной окружности.
Из~вершины~$A$ проведена высота~$A A_1$.
Докажите, что углы $B A A_1$ и~$O A C$ равны.
\\
\subproblem
Докажите, что \emph{ортоцентр} (точка пересечения высот) $H$
треугольника $ABC$, отраженный относительно стороны~$BC$, попадает в~точку
на~описанной окружности треугольника $ABC$.
\\
\subproblem
Докажите, что ортоцентр~$H$, отраженный относительно середины~$BC$, попадает
в~точку на~описанной окружности треугольника $ABC$, причем диаметрально
противоположную $A$.

\end{problems}
\resetproblem

\begin{problems}

\item
Две окружности пересекаются в~точках $M$ и~$K$.
Через $M$ и~$K$ проведены прямые $AB$ и~$CD$ соответственно, пересекающие
первую окружность в~точках $A$ и~$C$, вторую в~точках $B$ и~$D$.
Докажите, что $AC \parallel BD$.

\item
Внутри треугольника $ABC$ взята точка~$P$ так, что
$\angle BPC = \angle A + 60^{\circ}$,
$\angle APC = \angle B + 60^{\circ}$,
$\angle APB = \angle C + 60^{\circ}$.
Прямые $AP$, $BP$ и~$CP$ пересекают описанную окружность треугольника $ABC$
в~точках $A_1$, $B_1$ и~$C_1$ соответственно.
Докажите, что треугольник $A_1 B_1 C_1$~--- правильный.

\item
Окружности $S_1$ и~$S_2$ пересекаются в~точке~$A$.
Через точку~$A$ проведена прямая, пересекающая $S_1$ в~точке~$B$, а~$S_2$
в~точке~$C$.
В~точках $C$ и~$B$ проведены касательные к~окружностям, пересекающиеся
в~точке~$D$.
Докажите, что угол $\angle BDC$ не~зависит от~выбора прямой, проходящей
через $A$.

\item
Внутри квадрата $ABCD$ отметили точку~$E$ так, что
$\angle ECD = \angle EAC = \alpha$.
Найдите $\angle ABE$.

\item
$ABCD$~--- вписанный четырехугольник.
Лучи $AB$, $DC$ пересекаются в~точке~$P$, лучи $BC$, $AD$~--- в~точке~$Q$.
Докажите, что биссектрисы углов $APD$ и~$AQB$ перпендикулярны.

\item
$ABCD$~--- вписанный четырехугольник, $K$~--- середина дуги~$AB$,
не~содержащей точек $C$ и~$D$.
$P$ и~$Q$~--- точки пересечения пар хорд $CK$ и~$AB$, $DK$ и~$AB$
соответственно.
Докажите, что четырехугольник $CPQD$~--- вписанный.

\item
Точки $A$, $B$, $M$, $N$ лежат на~окружности в~указанном порядке.
Пусть $A_1$, $B_1$~--- такие точки на~окружности, что
$N A \perp M B_1$, $N B \perp M A_1$.
Докажите, что $A A_1 \parallel B B_1$.

\item
$O$~--- центр описанной окружности равнобокой трапеции $ABCD$ с~боковой
стороной $AB$, а~$K$~--- точка пересечения ее диагоналей.
Докажите, что точки $A$, $B$, $K$, $O$ лежат на~одной окружности.

\end{problems}

