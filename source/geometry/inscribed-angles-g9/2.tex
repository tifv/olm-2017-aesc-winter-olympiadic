% $date: 2017-01-10
% $timetable:
%   g9:
%     2017-01-10:
%       1:

% $matter[-contained,no-header]:
% - verbatim: \section*{Вписанные углы}\vspace{-1ex}
% - verbatim: \setproblem{8}
% - .[contained]

% $matter[contained,no-header,add-toc]:
% - .[-add-toc]

\subsection*{Добавка}

% $authors:
% # Алексей Александрович Пономарёв
% - Юлий Васильевич Тихонов

\begin{problems}

\item
Две окружности пересекаются в~точках $A$, $B$;
на~первой из~этих окружностей отмечена точка~$C$.
Прямые $CA$, $CB$ пересекают вторую окружность вторично в~точках $E$, $F$.
Докажите, что касательная к~первой окружности, восстановленная в~точке~$C$,
параллельна прямой~$EF$.

\item
Точки $A$, $B$, $C$, $D$ лежат на~одной окружности в~указанном порядке.
$K$, $L$, $M$, $N$~--- середины дуг $AB$, $BC$, $CD$, $DA$, не~содержащих
внутри четырех исходных точек.
Докажите, что $KM \perp LN$.

\item
В~треугольнике $ABC$ проведены медианы $B B_1$, $C C_1$.
Оказалось, что $\angle A B B_1 = \angle A C C_1$.
Докажите, что $AB = AC$.

\item
В~треугольнике $ABC$ проведена биссектриса~$AL$.
$K$~--- такая точка на~отрезке~$AC$, что $\angle BAC = \angle CLK$.
Докажите, что $BL = LK$.

\item
В~неравнобедренном треугольнике $ABC$ проведена биссектриса~$AD$.
$E$~--- точка пересечения прямой~$BC$ с~касательной к~описанной окружности
исходного треугольника, восстановленной в~точке~$A$.
Докажите, что $AE = DE$.

\end{problems}

%\item\emph{Лемма о~трилистнике, она~же лемма о~трезубце.}
%Пусть $I$~--- точка пересечения биссектрис треугольника $ABC$,
%$A_0$~--- середина дуги~$BC$ описанной окружности треугольника $ABC$,
%не~содержащей точку~$A$.
%Докажите, что $A_0 B = A_0 I = A_0 C$.

