% $date: 2017-01-11
% $timetable:
%   g1011r2:
%     2017-01-11:
%       3:

\section*{Счёт касательных}

% $authors:
% - Алексей Вадимович Доледенок

\claim{Напоминание}
В~треугольнике $ABC$ вписанная и~вневписанная окружности касается стороны~$BC$
в~точках $A_1$ и~$A_2$ соответственно.
Тогда $B A_1 = C A_2 = p - b$.

\begin{problems}

\item
Вневписанные окружности треугольника $ABC$ касаются продолжений стороны~$AB$
за~точки $A$ и~$B$ в~точках $X$ и~$Y$ соответственно.
Докажите, что середины отрезков $XY$ и~$AB$ совпадают.

\item
\subproblem
На~стороне~$BC$ треугольника $ABC$ выбрана точка~$D$.
В~треугольники $ABD$ и~$ACD$ вписаны окружности $\omega_1$ и~$\omega_2$,
которые касаются стороны~$AD$ в~точках $X$ и~$Y$ соответственно.
Выразите длину отрезка~$XY$ через длины отрезков $AB$, $AC$, $BD$ и~$CD$.
\\
\subproblem
\jeolmlabel{/geometry/tangents-counting-g1011r2/:2b}%
Пусть $A_1$~--- точка касания вписанной в~треугольник $ABC$ окружности
со~стороной~$BC$.
Докажите, что $\omega_1$ и~$\omega_2$ касаются тогда и~только тогда, когда
точки $D$ и~$A_1$ совпадают.
\\
\subproblem
Пусть $P$ и~$Q$~--- точки касания $\omega_1$ и~$\omega_2$ с~прямой~$BC$.
Докажите, что $PD = A_1 Q$.
%\\
%\subproblem
%Сформулируйте и~докажите утверждение аналогичное пункту
%\jeolmsubref{/geometry/tangents-counting-g1011r2/:criteria}
%для вневписанных окружностей треугольников $ABD$ и~$ACD$.

\item
В~выпуклом четырехугольнике $ABCD$ окружности, вписанные в~треугольники $ABC$
и~$ADC$, касаются.
Докажите, что окружности, вписанные в~треугольники $BCD$ и~$BAD$, тоже
касаются.

\item
Три попарно непересекающиеся окружности
$\omega_{x}$, $\omega_{y}$, $\omega_{z}$ радиусов $r_{x}$, $r_{y}$, $r_{z}$
соответственно лежат по~одну сторону относительно прямой~$t$ и~касаются ее
в~точках $X$, $Y$, $Z$ соответственно.
Известно, что $Y$~--- середина отрезка~$XZ$, $r_{x} = r_{z} = r$,
а~$r_{y} > r$.
Пусть $p$~--- одна из~общих внутренних касательных
к~окружностям~$\omega_x$ и~$\omega_y$, а~$q$~--- одна из~общих внутренних
касательных к~окружностям $\omega_y$ и~$\omega_z$.
В~пересечении прямых $p$, $q$, $t$ образовался неравнобедренный треугольник.
Докажите, что радиус его вписанной окружности равен $r$.

\item
В~произвольном треугольнике $ABC$ на~прямых $AB$ и~$AC$ отложим от~точки~$A$
вовне треугольника отрезки, равные $BC$.
Концы этих отрезков обозначим $A_1$ и~$A_2$.
Аналогично определим точки $B_1$, $B_2$, $C_1$, $C_2$.
Докажите, что эти шесть точек лежат на~одной окружности.

\end{problems}

