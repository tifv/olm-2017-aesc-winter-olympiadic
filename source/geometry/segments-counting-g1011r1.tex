% $date: 2017-01-10
% $timetable:
%   g1011r1:
%     2017-01-10:
%       2:

\section*{Счёт отрезочков}

% $authors:
% - Алексей Вадимович Доледенок

\begin{problems}

\item
Из~точки~$T$ к~окружности провели касательную~$TA$ и~секущую, пересекающую
окружность в~точках $B$ и~$C$.
Биссектриса угла $ATC$ пересекает хорды $AB$ и~$AC$ в~точках $P$ и~$Q$
соответственно.
Докажите, что $PA^2 = PB \cdot QC$.

\item
Дана равнобокая трапеция $ABCD$ ($AD \parallel BC$).
На~дуге~$AD$ (не~содержащей точек $B$ и~$C$) описанной окружности этой трапеции
произвольно выбрана точка~$M$.
Докажите, что основания перпендикуляров, опущенных из~вершин $A$ и~$D$
на~отрезки $BM$ и~$CM$, лежат на~одной окружности.

\item
На~прямых, содержащих высоты $B B_1$ и~$C C_1$ остроугольного
треугольника $ABC$, отметили точки, из~которых соответствующие стороны
(т.\,е. $AC$ и~$AB$ соответственно) видны под прямым углом.
Докажите, что четыре отмеченные точки лежат на~одной окружности.

%\item
%Три попарно непересекающиеся окружности
%$\omega_{x}$, $\omega_{y}$, $\omega_{z}$ радиусов $r_{x}$, $r_{y}$, $r_{z}$
%соответственно лежат по~одну сторону относительно прямой~$t$ и~касаются ее
%в~точках $X$, $Y$, $Z$ соответственно.
%Известно, что $Y$~--- середина отрезка~$XZ$, $r_{x} = r_{z} = r$,
%а~$r_{y} > r$.
%Пусть $p$~--- одна из~общих внутренних касательных
%к~окружностям~$\omega_x$ и~$\omega_y$, а~$q$~--- одна из~общих внутренних
%касательных к~окружностям $\omega_y$ и~$\omega_z$.
%В~пересечении прямых $p$, $q$, $t$ образовался неравнобедренный треугольник.
%Докажите, что радиус его вписанной окружности равен $r$.

\item
Точка~$I$~--- центр вписанной окружности треугольника $ABC$.
Точка~$M$ на~прямой~$BC$ такова, что $MI \perp AI$.
Точка~$D$~--- основание перпендикуляра из~$I$ на~$AM$.
Докажите, что точки $A$, $B$, $C$, $D$ лежат на~одной окружности.

\item
В~произвольном треугольнике $ABC$ на~прямых $AB$ и~$AC$ отложим от~точки~$A$
вовне треугольника отрезки, равные $BC$.
Концы этих отрезков обозначим $A_1$ и~$A_2$.
Аналогично определим точки $B_1$, $B_2$, $C_1$, $C_2$.
Докажите, что эти шесть точек лежат на~одной окружности.

\item
Прямые, касающиеся окружности~$\omega$ в~точках $B$ и~$D$, пересекаются
в~точке~$P$.
Прямая, проходящая через $P$, высекает на~окружности хорду~$AC$.
Через произвольную точку отрезка~$AC$ проведена прямая, параллельная $BD$.
Докажите, что она делит длины ломаных $ABC$ и~$ADC$ в~одинаковых отношениях.

\item
Продолжения медиан $A A_1$, $B B_1$ и~$C C_1$ треугольника $ABC$ пересекают его
описанную окружность в~точках $A_0$, $B_0$ и~$C_0$ соответственно.
Оказалось, что площади треугольников $A B C_0$, $A B_0 C$ и~$A_0 B C$ равны.
Докажите, что треугольник $ABC$ равносторонний.

\item
В~трапеции $ABCD$ боковая сторона~$CD$ перпендикулярна основаниям,
$O$~--- точка пересечения диагоналей.
На~описанной окружности треугольника $OCD$ взята точка~$S$, диаметрально
противоположная точке~$O$.
Докажите, что $\angle BSC = \angle ASD$.

\end{problems}

