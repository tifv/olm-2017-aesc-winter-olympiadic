% $date: 2017-01-11
% $timetable:
%   g9:
%     2017-01-11:
%       3:

\section*{Немного о трезубце}

% $authors:
% - Юлий Васильевич Тихонов

В~треугольнике $ABC$ биссектрисы углов $A$, $B$ и~$C$ пересекают описанную
окружность треугольника $ABC$ в~точках $A_0$, $B_0$ и~$C_0$ соответственно.
$I$~--- центр вписанной окружности.

\begin{problems}

\item\claim{Лемма о~трезубце}
Точка~$B_0$ равноудалена от~$A$, $C$ и~$I$.

\item
Докажите, что $A_0 C_0$~--- серединный перпендикуляр к~отрезку~$BI$.

\item
Обозначим точки пересечения $A_0 C_0$ с~$BA$ и~$BC$ через $X$ и~$Y$
соответственно.
Докажите, что $BXIY$~--- ромб.

\item
Обозначим через $P$ точку пересечения прямой, проходящей через точку~$I$
параллельно $AC$, и~прямой~$A_0 C_0$.
Докажите, что $PB$ касается описанной окружности треугольника $ABC$.

\item
Точка~$I$~--- центр вписанной окружности треугольника $ABC$.
Внутри треугольника выбрана точка~$P$ такая,
что
\[
    \angle PBA + \angle PCA
=
    \angle PBC + \angle PCB
\]
Докажите, что $AP \geq AI$, причем равенство выполняется тогда и~только тогда,
когда $P$ совпадает с~$I$.

\end{problems}

