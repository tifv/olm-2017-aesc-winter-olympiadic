% $date: 2017-01-09
% $timetable:
%   g9:
%     2017-01-09:
%       2:

\section*{Теория чисел}

% $authors:
% - Виктория Владимировна Журавлёва

\begin{problems}

\item
Число~$x$ таково, что среди четырех чисел
\[
    a = x - \sqrt{2}
, \quad
    b = x - 1 / x
, \quad
    c = x + 1 / x
, \quad
    d = x^2 + 2 \sqrt{2}
\]
ровно одно не~является целым.
Найдите все такие $x$.

\item
Даны $111$ различных натуральных чисел, не~превосходящих $500$.
Могло~ли оказаться, что для каждого из~этих чисел его последняя цифра совпадает
с~последней цифрой суммы всех остальных чисел?

\item
Натуральное число~$m$ таково, что сумма цифр в~десятичной записи числа~$8^m$
равна $8$.
Может~ли при этом последняя цифра числа~$8^m$ быть равной $6$?

\item
Найдите все такие числа~$a$, что для любого натурального~$n$ число
$a n (n + 2) (n + 3) (n + 4)$ будет целым.

\item
Найдите все такие тройки простых чисел $p, q, r$, что четвертая степень каждого
из~них, уменьшенная на~$1$, делится на~произведение остальных.

\item
Петя выбрал несколько последовательных натуральных чисел и~каждое записал либо
красным, либо синим карандашом (оба цвета присутствуют).
Может~ли сумма наименьшего общего кратного всех красных чисел и~наименьшего
общего кратного всех синих чисел являться степенью двойки?

\item
По~кругу стоят $10^{1000}$ натуральных чисел.
Между каждыми двумя соседними числами записали их наименьшее общее кратное.
Могут~ли эти наименьшие общие кратные образовать $10^{1000}$ последовательных
чисел (расположенных в~каком-то порядке)?

\end{problems}

